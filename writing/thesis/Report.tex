%% The report to gather all sections into, set formatting and compile.

\documentclass[a4paper, hidelinks]{article} %hidelinks removes the red boxes around hyperlinks

%%%%%%%%%% Packages %%%%%%%%%%

\usepackage{epstopdf} % to allow use of crest in title page
\usepackage{graphicx}
\usepackage[mathlines]{lineno}%line numbers
%\usepackage{natbib}
\usepackage[backend=biber,style=authoryear-comp]{biblatex}
\usepackage{hyperref} % for URLs
\usepackage[table, dvipsnames]{xcolor}
%%%%%%%%%% Custome Commands %%%%%%%%%%
\newcommand{\crest}{\includegraphics[width = 4cm, keepaspectratio]{../images/IC_Crest.eps}} % Imperial crest
\newcommand\wordcount{%% The report to gather all sections into, set formatting and compile.

\documentclass[a4paper, hidelinks]{article} %hidelinks removes the red boxes around hyperlinks

%%%%%%%%%% Packages %%%%%%%%%%

\usepackage{epstopdf} % to allow use of crest in title page
\usepackage{graphicx}
\usepackage[mathlines]{lineno}%line numbers
%\usepackage{natbib}
\usepackage[backend=biber,style=authoryear-comp]{biblatex}
\usepackage{hyperref} % for URLs
\usepackage[table, dvipsnames]{xcolor}
%%%%%%%%%% Custome Commands %%%%%%%%%%
\newcommand{\crest}{\includegraphics[width = 4cm, keepaspectratio]{../images/IC_Crest.eps}} % Imperial crest
\newcommand\wordcount{%% The report to gather all sections into, set formatting and compile.

\documentclass[a4paper, hidelinks]{article} %hidelinks removes the red boxes around hyperlinks

%%%%%%%%%% Packages %%%%%%%%%%

\usepackage{epstopdf} % to allow use of crest in title page
\usepackage{graphicx}
\usepackage[mathlines]{lineno}%line numbers
%\usepackage{natbib}
\usepackage[backend=biber,style=authoryear-comp]{biblatex}
\usepackage{hyperref} % for URLs
\usepackage[table, dvipsnames]{xcolor}
%%%%%%%%%% Custome Commands %%%%%%%%%%
\newcommand{\crest}{\includegraphics[width = 4cm, keepaspectratio]{../images/IC_Crest.eps}} % Imperial crest
\newcommand\wordcount{%% The report to gather all sections into, set formatting and compile.

\documentclass[a4paper, hidelinks]{article} %hidelinks removes the red boxes around hyperlinks

%%%%%%%%%% Packages %%%%%%%%%%

\usepackage{epstopdf} % to allow use of crest in title page
\usepackage{graphicx}
\usepackage[mathlines]{lineno}%line numbers
%\usepackage{natbib}
\usepackage[backend=biber,style=authoryear-comp]{biblatex}
\usepackage{hyperref} % for URLs
\usepackage[table, dvipsnames]{xcolor}
%%%%%%%%%% Custome Commands %%%%%%%%%%
\newcommand{\crest}{\includegraphics[width = 4cm, keepaspectratio]{../images/IC_Crest.eps}} % Imperial crest
\newcommand\wordcount{\input{Report.sum}} % to include word count 

%%%%%%%%%% Formatting %%%%%%%%%%
\usepackage[margin=2cm]{geometry} % margins of 2cm
\linespread{1.5} %1.5 spacing
%\renewcommand{\familydefault}{\sfdefault} % set font to arial clone (helvet)

\usepackage[compact]{titlesec} % reduce spacing bewteen section titles

%%%%%%%%%% Bibliography %%%%%%%%%%
\addbibresource{../Masters_Thesis.bib}


%%%%%%%%%% Document %%%%%%%%%%
\begin{document}
	%%%%%%%%%% Title Page %%%%%%%%%%
	\include{title_page}

	%%%%%%%%%% Abstract %%%%%%%%%%
	\section*{Abstract}
	\linenumbers

	
	
	\nolinenumbers
	%%%%%%%%%% Acknowledgements %%%%%%%%%%
	\include{Acknowledgements}
	
	%%%%%%%%%% Table of Contents %%%%%%%%%%
	\tableofcontents
	\newpage
	%% start line numbering


	%%%%%%%%%% Introduction %%%%%%%%%%
\section{Introduction}
	\linenumbers
	% Why is growth important
	Understanding how organisms grow and what factors play a role in determining growth
	Larger fish produce more offspring % this is known in general isometrically
	and may even lead to more offspring than if the same mass were spread over two fish \parencite{Barneche2018}.
	Larger fish also use energy more efficiently than multiple smaller ones per unit mass (because they have a lower mass specific metabolic rate \parencite{Peters1983}) so it may actually reduce stress upon the ecosystem from an energetic perspective which could be useful when trying to manage fish stocks %need citations for this
	It is already known that metabolic rates and the size of fish is dependant on temperature and with global warming understanding in greater detail how increased metabolic rates may affect growth is useful.
	
	%
	
	Growth is in essence a balancing act between how much energy an organism can acquire and the amount of energy required for maintenance, that is movement, digestion et cetera.  
	
	%power laws -- needs citation
	Key to understanding rates and their relationship with mass is the concept of power laws. Many biological traits can be described as scaling to the power of some other biological trait.  That is some rate, $Y$, can be expressed for any mass by $Y = Y_0 m^\beta$.  Power laws can be broadly categorised based on the value of their exponent, $\beta$.  Where the exponent does not equal one, the relationship is said to be allometric.  That is the trait does not increase at the same rate as the trait being compared against.  Where the exponent equals one the relationship is described as isometric, that is the two traits increase at the same rate. For the purposes of describing growth these relationships are indispensable.   
	
	Traditionally ontogenetic growth models have relied on knowing how large an organism is expected to grow.
	%	von Bertalanffy / puetter model
	The von Bertalanffy growth equation relies on knowing the longest a fish can be and the length of the fish at the beginning of measurements \parencite{vonBertalanffy1938, Putter1918}.  From here for a known growth rate, the length of the fish after a set amount of time has passed can be predicted.
	%	introduce west 2001 \cite{West2001}
		% then take it to energetics
	One of the best known examples of an OGM is the model developed by \cite{West2001}.  This model is parametrised around the average energy content of tissue asymptotic mass.  Asymptotic mass being the mass at which growth has essentially stopped due to metabolic cost and energy intake equalling each other.  
		%	improved by hou et al \cite{Hou2008} talk about including SDA
		%	
		%	move to including reproduction with \cite{Charnov2001}
	
		% bring up the caveats/assumptions of the model i.e. isometric metabolism and repro, 0.75 scaling of intake. optimal intake at all times etc. test
	% Discuss allometry and isometry here to highlight what scaling super or sub linearly means
	
	In classic OGMs this is most notable in determining how much energy an organism acquires at a given mass where gain = constant $\times$ mass$^{0.75}$.  This scaling is based on the work of \cite{West1997}
	
	In the face two of the key assumptions of \cite{Charnov2001, West2001} OGM not being true, that reproduction and metabolism scale isometrically, there is a need to take a novel approach to modelling fish growth, in particular choosing to focus on developing how intake is described to better reflect the real world.  To do this an obvious starting point is to model intake as a functional response \parencite{Holling1959} so as to reflect real world intake rates.
	% describe a functional response and the work behind it on a general level without equations.
	
	% samraats works feels like it belongs in the methods section more than intro since i have to delve into responses for it to make sense imo
	%	In order to parametrise the functional response for realistic  \cite{Pawar2012} proved invaluable in describing the 
%\subsection{section layout (old)}
%	What is an OGM and ontogenic growth?
%	
%	An ontogenetic model or OGM is a model which describes the development of an organism.  Within a metabolic framework this describes how an organism's assess to and and alocation of energy changes throughout its lifetime.
%	
%	explain west's model
%	
%	explain terms 
%	
%	why is `a` resting metabolic rate?
%	
%	explain charnov's contribution
%	
%	- graph with generic example of the plot for illustration purposes
%	
%	explain what I am adding to `cm` and why --> maybe leave for the methods?
%	
%	talk about barneche 2018
%	
%	raises question of scaling elsewhere 
%	
%	talk about samraats work and how I plan to use it
%	
%	the goal of my paper
	
	\nolinenumbers
	%%%%%%%%%% Methods %%%%%%%%%%
\section{Methods}
	\linenumbers
%	\subsection{Gain}
%	
%	\subsection{Loss}
%	\subsubsection{Maintenance Cost}
%	\subsubsection{Reproductive Cost}
%	\subsection{Reproductive Output}
%	
%	\subsection{notes}
%	c bounded between 0 and 1 since it is basically GSI
	\nolinenumbers
	%%%%%%%%%% Results %%%%%%%%%%
\section{Results}
	\linenumbers
	\subsection{tables and figures}
	
	\begin{centering}

		{\rowcolors{1}{green!80!yellow!50}{red!220!green!220!yellow!220)}}
			
		\begin{table}[h!]
			\label{parameters}
			\caption{Table describing parameters used in the model, along with values units and sources where applicable.}
			\begin{tabular}{c l l l l l}
				\hline
				Parameter 	& Description 			& Value 	& Units 	& Range 		& Source \\
				\hline
				$m$			& Mass					& ?			& kg day$^{-1}$& -			&		\\
				
				$B_m$		& Metabolic Cost		& -			& kg day$^{-1}$& - 			& \cite{Peters1983}\\
				
				$\alpha$	& Age of maturity		& -     	& day		& -				& -\\
				$c$			& Reproduction scaling constant & - & kg day$^{-1}$& 0-1 			& -\\
				$\rho$		& Reproduction scaling exponent	& -	&	-		& 0-1.5			& -\\
				$Z$			& Rate of instantaneous mortality& $2/\alpha$	& & & \cite{Charnov2001}\\
				$k$			& Reproductive senescence & 0.01\\
				
				$\epsilon$	& Resource Conversion Efficiency & 0.70 & - & - 		& \cite{Peters1983} \\
				$X_r$ 		& Resource Density		& -		& kg		& ?				& -\\
				$\gamma$	& Search rate scaling exponent & 0.68 in 2D	& - & - & \cite{Pawar2012} \\
							&						& 1.05 in 3D\\
				$a_0$		& Search rate scaling constant & $10^{-3.08}$ in 2D & m$^2$ s$^{-1}$ kg$^{-0.68}$   & &\cite{Pawar2012}	\\
							&						& $10^{-1.77}$ in 3D& m$^2$ s$^{-1}$ kg$^{-1.05} $\\
				$\beta$		& Handling time scaling exponent& 0.75 & - & - & \cite{Pawar2012}\\
				$t_{h, 0}$	& Handling time scaling constant& $10^{3.95}$ in 2D &kg$^{1-\beta}$ s& -& \cite{Pawar2012}	\\
							&						& $10^{3.04}$ in 3D			&kg$^{1-\beta}$ s\\
				\hline
			\end{tabular}
		\end{table}
	\end{centering}

	\nolinenumbers
	
	%%%%%%%%%% Discussion %%%%%%%%%%
\section{Discussion}
	\linenumbers
	
	\nolinenumbers
	
	%%%%%%%%%% Conclusion %%%%%%%%%%
\section{Conclusion}
	\linenumbers
	
	\nolinenumbers
	
	%%%%%%%%%% Bibliography %%%%%%%%%%
	\newpage
	
	\addcontentsline{toc}{section}{\protect\numberline{}References}
	
	\linenumbers
	\printbibliography
	\nolinenumbers
	
	%%%%%%%%%% SI %%%%%%%%%%
	\newpage
	\include{SI}
\end{document}} % to include word count 

%%%%%%%%%% Formatting %%%%%%%%%%
\usepackage[margin=2cm]{geometry} % margins of 2cm
\linespread{1.5} %1.5 spacing
%\renewcommand{\familydefault}{\sfdefault} % set font to arial clone (helvet)

\usepackage[compact]{titlesec} % reduce spacing bewteen section titles

%%%%%%%%%% Bibliography %%%%%%%%%%
\addbibresource{../Masters_Thesis.bib}


%%%%%%%%%% Document %%%%%%%%%%
\begin{document}
	%%%%%%%%%% Title Page %%%%%%%%%%
	%\newcommand{\crest}{\includegraphics[width = 4cm, keepaspectratio]{../images/IC_Crest.eps}} % Imperial crest
%%formating

\begin{titlepage} % Suppresses headers and footers on the title page
	\includegraphics[width = 7cm, keepaspectratio, left]{../images/imperial_logo}
	\centering % Centre everything on the title page
	
	\scshape % Use small caps for all text on the title page
	
%	\vspace*{\baselineskip} % White space at the top of the page
	
	%------------------------------------------------
	%	Title
	%------------------------------------------------
	
	\rule{\textwidth}{1.6pt}\vspace*{-\baselineskip}\vspace*{2pt} % Thick horizontal rule
	\rule{\textwidth}{0.4pt} % Thin horizontal rule
	
	\vspace{0.75\baselineskip} % Whitespace above the title
	
	{\LARGE The role of resource supply in shaping ontogenetic growth and allocation  in fish\\} % Title
	
	\vspace{0.75\baselineskip} % Whitespace below the title
	
	\rule{\textwidth}{0.4pt}\vspace*{-\baselineskip}\vspace{3.2pt} % Thin horizontal rule
	\rule{\textwidth}{1.6pt} % Thick horizontal rule
	
	\vspace{1\baselineskip} % Whitespace after the title block
	
	%------------------------------------------------
	%	Subtitle
	%------------------------------------------------
	
	%SUBTITLE? % Subtitle or further description
%	Student:
	
	
	\vspace{0.5\baselineskip} % Whitespace before 
	
	{\scshape\Large D\'onal Burns  \\} % my name
	
	\vspace{0.5\baselineskip} % Whitespace below 
	
	\textit{CID: 01749638 \\ Imperial College London \\ Email: donal.burns@imperial.ac.uk} % affiliation and email
	
	\vspace*{2\baselineskip} % Whitespace under the subtitle
	
	
	
%	Supervisor:
%	
%	
%	\vspace{0.5\baselineskip} % Whitespace before 
%	
%	{\scshape\Large Samraat Pawar \\} % supervisor name
%	
%	\vspace{0.5\baselineskip} % Whitespace below 
%	
%	\textit{Imperial College London \\ Email: s.pawar@imperial.ac.uk} % affiliation and email
%	
%	\vspace{3cm} % Whitespace between 
	

	
	%% crest
	
%	\includegraphics[width = 4cm, keepaspectratio]{../images/IC_crest.pdf}
	
	\vspace{0.3\baselineskip} % Whitespace under the Uni logo
	
	Submitted: August 27$^{th}$ 2020 % Publication Date
	
	

		%%Submission clause
	\vspace{3cm}
	
	A thesis submitted in partial fulfilment of the requirements for the degree of
	%Computational Methods in Ecology and Evolution 
	Master of Science at Imperial College London
	\vspace{0.5\baselineskip}
	
	Formatted in the journal style of Functional Ecology	
	\vspace{0.5\baselineskip}
	
	Submitted for the MSc in Computational Methods in Ecology and Evolution
	
\end{titlepage}


	%%%%%%%%%% Abstract %%%%%%%%%%
	\section*{Abstract}
	\linenumbers

	
	
	\nolinenumbers
	%%%%%%%%%% Acknowledgements %%%%%%%%%%
	%\thispagestyle{empty}

\mbox{}\newline\vspace{10mm} \mbox{}\LARGE
%
{\bf Acknowledgements} \normalsize \vspace{5mm}\\
I would like to thank my supervisor Dr. Samraat Pawar as well as fellow lab members Tom Clegg and Olivia Moris for giving me so much of their time on weekly, and on occasion more than weekly, basis.  I would also like to thank Dr. Diego Barneche for his invaluable feedback and Dr. Van Savage for his assistance with some of the initial model development.






	
	%%%%%%%%%% Table of Contents %%%%%%%%%%
	\tableofcontents
	\newpage
	%% start line numbering


	%%%%%%%%%% Introduction %%%%%%%%%%
\section{Introduction}
	\linenumbers
	% Why is growth important
	Understanding how organisms grow and what factors play a role in determining growth
	Larger fish produce more offspring % this is known in general isometrically
	and may even lead to more offspring than if the same mass were spread over two fish \parencite{Barneche2018}.
	Larger fish also use energy more efficiently than multiple smaller ones per unit mass (because they have a lower mass specific metabolic rate \parencite{Peters1983}) so it may actually reduce stress upon the ecosystem from an energetic perspective which could be useful when trying to manage fish stocks %need citations for this
	It is already known that metabolic rates and the size of fish is dependant on temperature and with global warming understanding in greater detail how increased metabolic rates may affect growth is useful.
	
	%
	
	Growth is in essence a balancing act between how much energy an organism can acquire and the amount of energy required for maintenance, that is movement, digestion et cetera.  
	
	%power laws -- needs citation
	Key to understanding rates and their relationship with mass is the concept of power laws. Many biological traits can be described as scaling to the power of some other biological trait.  That is some rate, $Y$, can be expressed for any mass by $Y = Y_0 m^\beta$.  Power laws can be broadly categorised based on the value of their exponent, $\beta$.  Where the exponent does not equal one, the relationship is said to be allometric.  That is the trait does not increase at the same rate as the trait being compared against.  Where the exponent equals one the relationship is described as isometric, that is the two traits increase at the same rate. For the purposes of describing growth these relationships are indispensable.   
	
	Traditionally ontogenetic growth models have relied on knowing how large an organism is expected to grow.
	%	von Bertalanffy / puetter model
	The von Bertalanffy growth equation relies on knowing the longest a fish can be and the length of the fish at the beginning of measurements \parencite{vonBertalanffy1938, Putter1918}.  From here for a known growth rate, the length of the fish after a set amount of time has passed can be predicted.
	%	introduce west 2001 \cite{West2001}
		% then take it to energetics
	One of the best known examples of an OGM is the model developed by \cite{West2001}.  This model is parametrised around the average energy content of tissue asymptotic mass.  Asymptotic mass being the mass at which growth has essentially stopped due to metabolic cost and energy intake equalling each other.  
		%	improved by hou et al \cite{Hou2008} talk about including SDA
		%	
		%	move to including reproduction with \cite{Charnov2001}
	
		% bring up the caveats/assumptions of the model i.e. isometric metabolism and repro, 0.75 scaling of intake. optimal intake at all times etc. test
	% Discuss allometry and isometry here to highlight what scaling super or sub linearly means
	
	In classic OGMs this is most notable in determining how much energy an organism acquires at a given mass where gain = constant $\times$ mass$^{0.75}$.  This scaling is based on the work of \cite{West1997}
	
	In the face two of the key assumptions of \cite{Charnov2001, West2001} OGM not being true, that reproduction and metabolism scale isometrically, there is a need to take a novel approach to modelling fish growth, in particular choosing to focus on developing how intake is described to better reflect the real world.  To do this an obvious starting point is to model intake as a functional response \parencite{Holling1959} so as to reflect real world intake rates.
	% describe a functional response and the work behind it on a general level without equations.
	
	% samraats works feels like it belongs in the methods section more than intro since i have to delve into responses for it to make sense imo
	%	In order to parametrise the functional response for realistic  \cite{Pawar2012} proved invaluable in describing the 
%\subsection{section layout (old)}
%	What is an OGM and ontogenic growth?
%	
%	An ontogenetic model or OGM is a model which describes the development of an organism.  Within a metabolic framework this describes how an organism's assess to and and alocation of energy changes throughout its lifetime.
%	
%	explain west's model
%	
%	explain terms 
%	
%	why is `a` resting metabolic rate?
%	
%	explain charnov's contribution
%	
%	- graph with generic example of the plot for illustration purposes
%	
%	explain what I am adding to `cm` and why --> maybe leave for the methods?
%	
%	talk about barneche 2018
%	
%	raises question of scaling elsewhere 
%	
%	talk about samraats work and how I plan to use it
%	
%	the goal of my paper
	
	\nolinenumbers
	%%%%%%%%%% Methods %%%%%%%%%%
\section{Methods}
	\linenumbers
%	\subsection{Gain}
%	
%	\subsection{Loss}
%	\subsubsection{Maintenance Cost}
%	\subsubsection{Reproductive Cost}
%	\subsection{Reproductive Output}
%	
%	\subsection{notes}
%	c bounded between 0 and 1 since it is basically GSI
	\nolinenumbers
	%%%%%%%%%% Results %%%%%%%%%%
\section{Results}
	\linenumbers
	\subsection{tables and figures}
	
	\begin{centering}

		{\rowcolors{1}{green!80!yellow!50}{red!220!green!220!yellow!220)}}
			
		\begin{table}[h!]
			\label{parameters}
			\caption{Table describing parameters used in the model, along with values units and sources where applicable.}
			\begin{tabular}{c l l l l l}
				\hline
				Parameter 	& Description 			& Value 	& Units 	& Range 		& Source \\
				\hline
				$m$			& Mass					& ?			& kg day$^{-1}$& -			&		\\
				
				$B_m$		& Metabolic Cost		& -			& kg day$^{-1}$& - 			& \cite{Peters1983}\\
				
				$\alpha$	& Age of maturity		& -     	& day		& -				& -\\
				$c$			& Reproduction scaling constant & - & kg day$^{-1}$& 0-1 			& -\\
				$\rho$		& Reproduction scaling exponent	& -	&	-		& 0-1.5			& -\\
				$Z$			& Rate of instantaneous mortality& $2/\alpha$	& & & \cite{Charnov2001}\\
				$k$			& Reproductive senescence & 0.01\\
				
				$\epsilon$	& Resource Conversion Efficiency & 0.70 & - & - 		& \cite{Peters1983} \\
				$X_r$ 		& Resource Density		& -		& kg		& ?				& -\\
				$\gamma$	& Search rate scaling exponent & 0.68 in 2D	& - & - & \cite{Pawar2012} \\
							&						& 1.05 in 3D\\
				$a_0$		& Search rate scaling constant & $10^{-3.08}$ in 2D & m$^2$ s$^{-1}$ kg$^{-0.68}$   & &\cite{Pawar2012}	\\
							&						& $10^{-1.77}$ in 3D& m$^2$ s$^{-1}$ kg$^{-1.05} $\\
				$\beta$		& Handling time scaling exponent& 0.75 & - & - & \cite{Pawar2012}\\
				$t_{h, 0}$	& Handling time scaling constant& $10^{3.95}$ in 2D &kg$^{1-\beta}$ s& -& \cite{Pawar2012}	\\
							&						& $10^{3.04}$ in 3D			&kg$^{1-\beta}$ s\\
				\hline
			\end{tabular}
		\end{table}
	\end{centering}

	\nolinenumbers
	
	%%%%%%%%%% Discussion %%%%%%%%%%
\section{Discussion}
	\linenumbers
	
	\nolinenumbers
	
	%%%%%%%%%% Conclusion %%%%%%%%%%
\section{Conclusion}
	\linenumbers
	
	\nolinenumbers
	
	%%%%%%%%%% Bibliography %%%%%%%%%%
	\newpage
	
	\addcontentsline{toc}{section}{\protect\numberline{}References}
	
	\linenumbers
	\printbibliography
	\nolinenumbers
	
	%%%%%%%%%% SI %%%%%%%%%%
	\newpage
	
%to have table and figures numbered from 1 with prefix
\newcommand{\beginsupplement}{%
	\addcontentsline{toc}{section}{\protect\numberline{}Supplementary Information}
	\setcounter{table}{0}
	\renewcommand{\thetable}{S\arabic{table}}%
	\setcounter{figure}{0}
	\renewcommand{\thefigure}{S\arabic{figure}}%
	\setcounter{equation}{0}
	\renewcommand{\theequation}{S\arabic{equation}}
}

\begin{refsection} % for seperate bibliography to main text
\section*{Supplementary Information}
\beginsupplement
%\subsection*{notes}
%need section on value conversions and derivations
%
%move any unreferenced sensitivity analyses here.

%\subsection{Figures}
\subsection{Unit Conversions}
\subsubsection{Functional Response ($ f(\cdot) $)}
	\begin{align}
		\label{Fr_Conversion}
		kg \cdot s^{-1} \cdot 24 \cdot 60 \cdot 60 &= kg \cdot d{-1}
	\end{align}
	
\subsubsection{Metabolic Cost ($ B_m $)}
	Conversion factor for joules to kg wet mass from \citeauthor{weathers2012fundamentals} (\citeyear{weathers2012fundamentals}).
	\begin{align}
		\begin{split}
			\label{Bm_Conversion}
			J \cdot s^{-1} \cdot 24 \cdot 60 \cdot 60 &= J \cdot d{-1}\\
			J \cdot d{-1} \cdot 2.5 \times 10^{-4} &= kg \cdot d{-1}
		\end{split}
	\end{align}
	
\subsection{Growth Curves}
%TODO ensure these are mentioned in the text
	\begin{figure}[H]
	\centering 
	\includegraphics[width=0.7\textwidth]{../../results/pretty_curve}
	\caption{Example of the growth curve and cumulative reproduction expected from a traditional OGM model. Maturation occurs at 1000 days, after which growth is less steep until reaching asymptotic mass.  }
	\label{OGM_Curve}
\end{figure}

\begin{figure}[h]
	\centering
	\includegraphics[width=0.7\textwidth]{../../results/report_growth_curve.pdf}
	\caption{The growth over a fish which consumes in 2D.  Maturation occurs at 5 years (1825 days).  The fish was allowed to shrink by 5\% at the onset of reproduction.}
	\label{growth_curve}
\end{figure}

\clearpage
\subsection{Sensitivity Analysis}
\subsubsection{Maturation Time}
\begin{figure}[H]
	\centering
	\includegraphics[width=\linewidth]{../../results/Sens_High_Resources_Maturation_Time_short_exp1}
	\caption{Effect of maturation time on $c$ and $\rho$ where $\mu = 1$ and resource density is high (100 kg/m$^D$, where $D$ is the dimension).}
	\label{fig:senshighresourcesmaturationtimeshortexp1}
\end{figure}
\begin{figure}[h]
	\centering
	\includegraphics[width=\linewidth]{../../results/Sens_High_Resources_Maturation_Time_short_exp075}
	\caption{Effect of maturation time on $c$ and $\rho$ where $\mu = 0.75$ and resource density is high (100 kg/m$^D$, where $D$ is the dimension).}
	\label{fig:sensmaturationtimeshortexp075}
\end{figure}
\begin{figure}[h]
	\centering
	\includegraphics[width=\linewidth]{../../results/Sens_Low_Resources_Maturation_Time_short_exp1}
	\caption{Effect of maturation time on $c$ and $\rho$ where $\mu = 1$ and resource density is low (0.11 kg/m$^D$, where $D$ is the dimension).}
	\label{fig:senslowresourcesmaturationtimeshortexp1}
\end{figure}
\begin{figure}[h]
	\centering
	\includegraphics[width=\linewidth]{../../results/Sens_Low_Resources_Maturation_Time_short_exp075}
	\caption{Effect of maturation time on $c$ and $\rho$ where $\mu = 0.75$ and resource density is low(0.11 kg/m$^D$, where $D$ is the dimension).}
	\label{fig:senslowresourcesmaturationtimeshortexp075}
\end{figure}
\begin{figure}[h]
	\centering
	\includegraphics[width=\linewidth]{../../results/Sens_Very_Low_Resources_Maturation_Time_short_exp1}
	\caption{Effect of maturation time on $c$ and $\rho$ where $\mu = 1$ and resource density is very low (0.01 kg/m$^D$, where $D$ is the dimension).  At this resource density reproduction can only occur in 3D.}
	\label{fig:sensverylowresourcesmaturationtimeshortexp1}
\end{figure}
\begin{figure}[h]
	\centering
	\includegraphics[width=\linewidth]{../../results/Sens_Very_Low_Resources_Maturation_Time_short_exp075}
	\caption{Effect of maturation time on $c$ and $\rho$ where $\mu = 0.75$ and resource density is very low (0.01 kg/m$^D$, where $D$ is the dimension).  At this resource density reproduction can only occur in 3D.}
	\label{fig:sensverylowresourcesmaturationtimeshortexp075}
\end{figure}




\clearpage
\subsubsection{Metabolic Exponent ($\mu$)}
\begin{figure}[H]
	\centering
	\includegraphics[width=\linewidth]{../../results/Sens_High_Resources_Metabolic_Exponent}
	\caption{Effect of metabolic on $c$ and $\rho$ where resource density is high (100 kg/m$^D$, where $D$ is the dimension)}
	\label{fig:senshighresourcesmetabolicexponent}
\end{figure}
\begin{figure}[h]
	\centering
	\includegraphics[width=\linewidth]{../../results/Sens_Low_Resources_Metabolic_Exponent}
	\caption{Effect of metabolic on $c$ and $\rho$ where resource density is low (0.11 kg/m$^D$, where $D$ is the dimension)}
	\label{fig:senslowresourcesmetabolicexponent}
\end{figure}
\begin{figure}[h]
	\centering
	\includegraphics[width=\linewidth]{../../results/Sens_Very_Low_Resources_Metabolic_Exponent}
	\caption{Effect of metabolic on $c$ and $\rho$ where resource density is very low (0.01 kg/m$^D$, where $D$ is the dimension).  At this resource density reproduction can only occur in 3D.}
	\label{fig:sensverylowresourcesmetabolicexponent}
\end{figure}



\subsubsection{Resource Density}

\begin{figure}[H]
	\centering
	\includegraphics[width=\linewidth]{../../results/Sens_Resource_Density_exp1_broad}
	\caption{Effect of resource density on $c$ and $\rho$ where $\mu = 1$.  Over larger values for resource density.  3D quickly saturates at this density, thus is a nearly straight horizontal line.  See Fig. \ref{fig:sensresourcedensityexp075fine} for detail at lower resource density.  Units are $kg/m^D$, where $D$ is the dimension.}
	\label{fig:sensresourcedensityexp1broad}
\end{figure}
\begin{figure}[h]
	\centering
	\includegraphics[width=\linewidth]{../../results/Sens_Resource_Density_exp075_broad}
	\caption{Effect of resource density on $c$ and $\rho$ where $\mu = 0.75$.  Over larger values for resource density.  There is a lot of numeric instability across resource densities, but the trend appears to be somewhat stable around $\sim$0.8 in 2D and $\sim$0.53 in 3D See Fig. \ref{fig:sensresourcedensityexp075fine} for detail at lower resource density.  Units are $kg/m^D$, where $D$ is the dimension.}
	\label{fig:sensresourcedensityexp075broad}
\end{figure}
	\begin{figure}[h!]
	\centering
	\includegraphics[width=\linewidth]{../../results/Sens_Resource_Density_exp1_fine}
	\caption{Effect of resource density on $c$ and $\rho$ where $\mu = 1$.  Demonstrates the expected trend that under limiting resources the higher scaling of 3D search rate  allows for steeper reproductive scaling (Table \ref{parameters}).  As resources increase and supply shifts more towards being defined by the inverse of handling time, steeper scaling in 2D allows for higher $\rho$ values.  Units are kg/m$^D$, where $D$ is the dimension.}
	\label{fig:sensresourcedensityexp1fine}
\end{figure}





\subsubsection{Proportion of Shrinking Allowed}
\begin{figure}[H]
	\centering
	\includegraphics[width=\linewidth]{../../results/Sens_HighResources_Shrink_exp1}
	\caption{Effect of proportion of shrinking allowed on $c$ and $\rho$ where $\mu = 1$ and resource density is high (100 kg/m$^D$, where $D$ is the dimension).}
	\label{fig:senshighresourcesshrinkexp1}
\end{figure}
\begin{figure}[h]
	\centering
	\includegraphics[width=\linewidth]{../../results/Sens_HighResources_Shrink_exp075}
	\caption{Effect of proportion of shrinking allowed on $c$ and $\rho$ where $\mu = 0.75$ and resource density is high (100 kg/m$^D$, where $D$ is the dimension).}
	\label{fig:senshighresourcesshrinkexp075}
\end{figure}

\begin{figure}[h]
	\centering
	\includegraphics[width=\linewidth]{../../results/Sens_LowResources_Shrink_exp1}
	\caption{Effect of proportion of shrinking allowed on $c$ and $\rho$ where $\mu = 1$ and resource density is low (0.11 kg/m$^D$, where $D$ is the dimension)}
	\label{fig:senslowresourcesshrinkexp1}
\end{figure}
\begin{figure}[h]
	\centering
	\includegraphics[width=\linewidth]{../../results/Sens_LowResources_Shrink_exp075}
	\caption{Effect of proportion of shrinking allowed on $c$ and $\rho$ where $\mu = 0.75$ and resource density is low (0.11 kg/m$^D$, where $D$ is the dimension)}
	\label{fig:senslowresourcesshrinkexp075}
\end{figure}
\begin{figure}[h]
	\centering
	\includegraphics[width=\linewidth]{../../results/Sens_VeryLowResources_Shrink_exp1}
	\caption{Effect of proportion of shrinking allowed on $c$ and $\rho$ where $\mu = 1$ and resource density is very low (0.01 kg/m$^D$, where $D$ is the dimension).  The resource density only allows for reproduction to occur on 3D.}
	\label{fig:sensverylowresourcesshrinkexp1}
\end{figure}
\begin{figure}[h]
	\centering
	\includegraphics[width=\linewidth]{../../results/Sens_VeryLowResources_Shrink_exp075}
	\caption{Effect of proportion of shrinking allowed on $c$ and $\rho$ where $\mu = 0.75$ and resource density is very low (0.01 kg/m$^D$, where $D$ is the dimension).  The resource density only allows for reproduction to occur on 3D.}
	\label{fig:sensverylowresourcesshrinkexp075}
\end{figure}

\clearpage
	\printbibliography
\end{refsection}

\end{document}} % to include word count 

%%%%%%%%%% Formatting %%%%%%%%%%
\usepackage[margin=2cm]{geometry} % margins of 2cm
\linespread{1.5} %1.5 spacing
%\renewcommand{\familydefault}{\sfdefault} % set font to arial clone (helvet)

\usepackage[compact]{titlesec} % reduce spacing bewteen section titles

%%%%%%%%%% Bibliography %%%%%%%%%%
\addbibresource{../Masters_Thesis.bib}


%%%%%%%%%% Document %%%%%%%%%%
\begin{document}
	%%%%%%%%%% Title Page %%%%%%%%%%
	%\newcommand{\crest}{\includegraphics[width = 4cm, keepaspectratio]{../images/IC_Crest.eps}} % Imperial crest
%%formating

\begin{titlepage} % Suppresses headers and footers on the title page
	\includegraphics[width = 7cm, keepaspectratio, left]{../images/imperial_logo}
	\centering % Centre everything on the title page
	
	\scshape % Use small caps for all text on the title page
	
%	\vspace*{\baselineskip} % White space at the top of the page
	
	%------------------------------------------------
	%	Title
	%------------------------------------------------
	
	\rule{\textwidth}{1.6pt}\vspace*{-\baselineskip}\vspace*{2pt} % Thick horizontal rule
	\rule{\textwidth}{0.4pt} % Thin horizontal rule
	
	\vspace{0.75\baselineskip} % Whitespace above the title
	
	{\LARGE The role of resource supply in shaping ontogenetic growth and allocation  in fish\\} % Title
	
	\vspace{0.75\baselineskip} % Whitespace below the title
	
	\rule{\textwidth}{0.4pt}\vspace*{-\baselineskip}\vspace{3.2pt} % Thin horizontal rule
	\rule{\textwidth}{1.6pt} % Thick horizontal rule
	
	\vspace{1\baselineskip} % Whitespace after the title block
	
	%------------------------------------------------
	%	Subtitle
	%------------------------------------------------
	
	%SUBTITLE? % Subtitle or further description
%	Student:
	
	
	\vspace{0.5\baselineskip} % Whitespace before 
	
	{\scshape\Large D\'onal Burns  \\} % my name
	
	\vspace{0.5\baselineskip} % Whitespace below 
	
	\textit{CID: 01749638 \\ Imperial College London \\ Email: donal.burns@imperial.ac.uk} % affiliation and email
	
	\vspace*{2\baselineskip} % Whitespace under the subtitle
	
	
	
%	Supervisor:
%	
%	
%	\vspace{0.5\baselineskip} % Whitespace before 
%	
%	{\scshape\Large Samraat Pawar \\} % supervisor name
%	
%	\vspace{0.5\baselineskip} % Whitespace below 
%	
%	\textit{Imperial College London \\ Email: s.pawar@imperial.ac.uk} % affiliation and email
%	
%	\vspace{3cm} % Whitespace between 
	

	
	%% crest
	
%	\includegraphics[width = 4cm, keepaspectratio]{../images/IC_crest.pdf}
	
	\vspace{0.3\baselineskip} % Whitespace under the Uni logo
	
	Submitted: August 27$^{th}$ 2020 % Publication Date
	
	

		%%Submission clause
	\vspace{3cm}
	
	A thesis submitted in partial fulfilment of the requirements for the degree of
	%Computational Methods in Ecology and Evolution 
	Master of Science at Imperial College London
	\vspace{0.5\baselineskip}
	
	Formatted in the journal style of Functional Ecology	
	\vspace{0.5\baselineskip}
	
	Submitted for the MSc in Computational Methods in Ecology and Evolution
	
\end{titlepage}


	%%%%%%%%%% Abstract %%%%%%%%%%
	\section*{Abstract}
	\linenumbers

	
	
	\nolinenumbers
	%%%%%%%%%% Acknowledgements %%%%%%%%%%
	%\thispagestyle{empty}

\mbox{}\newline\vspace{10mm} \mbox{}\LARGE
%
{\bf Acknowledgements} \normalsize \vspace{5mm}\\
I would like to thank my supervisor Dr. Samraat Pawar as well as fellow lab members Tom Clegg and Olivia Moris for giving me so much of their time on weekly, and on occasion more than weekly, basis.  I would also like to thank Dr. Diego Barneche for his invaluable feedback and Dr. Van Savage for his assistance with some of the initial model development.






	
	%%%%%%%%%% Table of Contents %%%%%%%%%%
	\tableofcontents
	\newpage
	%% start line numbering


	%%%%%%%%%% Introduction %%%%%%%%%%
\section{Introduction}
	\linenumbers
	% Why is growth important
	Understanding how organisms grow and what factors play a role in determining growth
	Larger fish produce more offspring % this is known in general isometrically
	and may even lead to more offspring than if the same mass were spread over two fish \parencite{Barneche2018}.
	Larger fish also use energy more efficiently than multiple smaller ones per unit mass (because they have a lower mass specific metabolic rate \parencite{Peters1983}) so it may actually reduce stress upon the ecosystem from an energetic perspective which could be useful when trying to manage fish stocks %need citations for this
	It is already known that metabolic rates and the size of fish is dependant on temperature and with global warming understanding in greater detail how increased metabolic rates may affect growth is useful.
	
	%
	
	Growth is in essence a balancing act between how much energy an organism can acquire and the amount of energy required for maintenance, that is movement, digestion et cetera.  
	
	%power laws -- needs citation
	Key to understanding rates and their relationship with mass is the concept of power laws. Many biological traits can be described as scaling to the power of some other biological trait.  That is some rate, $Y$, can be expressed for any mass by $Y = Y_0 m^\beta$.  Power laws can be broadly categorised based on the value of their exponent, $\beta$.  Where the exponent does not equal one, the relationship is said to be allometric.  That is the trait does not increase at the same rate as the trait being compared against.  Where the exponent equals one the relationship is described as isometric, that is the two traits increase at the same rate. For the purposes of describing growth these relationships are indispensable.   
	
	Traditionally ontogenetic growth models have relied on knowing how large an organism is expected to grow.
	%	von Bertalanffy / puetter model
	The von Bertalanffy growth equation relies on knowing the longest a fish can be and the length of the fish at the beginning of measurements \parencite{vonBertalanffy1938, Putter1918}.  From here for a known growth rate, the length of the fish after a set amount of time has passed can be predicted.
	%	introduce west 2001 \cite{West2001}
		% then take it to energetics
	One of the best known examples of an OGM is the model developed by \cite{West2001}.  This model is parametrised around the average energy content of tissue asymptotic mass.  Asymptotic mass being the mass at which growth has essentially stopped due to metabolic cost and energy intake equalling each other.  
		%	improved by hou et al \cite{Hou2008} talk about including SDA
		%	
		%	move to including reproduction with \cite{Charnov2001}
	
		% bring up the caveats/assumptions of the model i.e. isometric metabolism and repro, 0.75 scaling of intake. optimal intake at all times etc. test
	% Discuss allometry and isometry here to highlight what scaling super or sub linearly means
	
	In classic OGMs this is most notable in determining how much energy an organism acquires at a given mass where gain = constant $\times$ mass$^{0.75}$.  This scaling is based on the work of \cite{West1997}
	
	In the face two of the key assumptions of \cite{Charnov2001, West2001} OGM not being true, that reproduction and metabolism scale isometrically, there is a need to take a novel approach to modelling fish growth, in particular choosing to focus on developing how intake is described to better reflect the real world.  To do this an obvious starting point is to model intake as a functional response \parencite{Holling1959} so as to reflect real world intake rates.
	% describe a functional response and the work behind it on a general level without equations.
	
	% samraats works feels like it belongs in the methods section more than intro since i have to delve into responses for it to make sense imo
	%	In order to parametrise the functional response for realistic  \cite{Pawar2012} proved invaluable in describing the 
%\subsection{section layout (old)}
%	What is an OGM and ontogenic growth?
%	
%	An ontogenetic model or OGM is a model which describes the development of an organism.  Within a metabolic framework this describes how an organism's assess to and and alocation of energy changes throughout its lifetime.
%	
%	explain west's model
%	
%	explain terms 
%	
%	why is `a` resting metabolic rate?
%	
%	explain charnov's contribution
%	
%	- graph with generic example of the plot for illustration purposes
%	
%	explain what I am adding to `cm` and why --> maybe leave for the methods?
%	
%	talk about barneche 2018
%	
%	raises question of scaling elsewhere 
%	
%	talk about samraats work and how I plan to use it
%	
%	the goal of my paper
	
	\nolinenumbers
	%%%%%%%%%% Methods %%%%%%%%%%
\section{Methods}
	\linenumbers
%	\subsection{Gain}
%	
%	\subsection{Loss}
%	\subsubsection{Maintenance Cost}
%	\subsubsection{Reproductive Cost}
%	\subsection{Reproductive Output}
%	
%	\subsection{notes}
%	c bounded between 0 and 1 since it is basically GSI
	\nolinenumbers
	%%%%%%%%%% Results %%%%%%%%%%
\section{Results}
	\linenumbers
	\subsection{tables and figures}
	
	\begin{centering}

		{\rowcolors{1}{green!80!yellow!50}{red!220!green!220!yellow!220)}}
			
		\begin{table}[h!]
			\label{parameters}
			\caption{Table describing parameters used in the model, along with values units and sources where applicable.}
			\begin{tabular}{c l l l l l}
				\hline
				Parameter 	& Description 			& Value 	& Units 	& Range 		& Source \\
				\hline
				$m$			& Mass					& ?			& kg day$^{-1}$& -			&		\\
				
				$B_m$		& Metabolic Cost		& -			& kg day$^{-1}$& - 			& \cite{Peters1983}\\
				
				$\alpha$	& Age of maturity		& -     	& day		& -				& -\\
				$c$			& Reproduction scaling constant & - & kg day$^{-1}$& 0-1 			& -\\
				$\rho$		& Reproduction scaling exponent	& -	&	-		& 0-1.5			& -\\
				$Z$			& Rate of instantaneous mortality& $2/\alpha$	& & & \cite{Charnov2001}\\
				$k$			& Reproductive senescence & 0.01\\
				
				$\epsilon$	& Resource Conversion Efficiency & 0.70 & - & - 		& \cite{Peters1983} \\
				$X_r$ 		& Resource Density		& -		& kg		& ?				& -\\
				$\gamma$	& Search rate scaling exponent & 0.68 in 2D	& - & - & \cite{Pawar2012} \\
							&						& 1.05 in 3D\\
				$a_0$		& Search rate scaling constant & $10^{-3.08}$ in 2D & m$^2$ s$^{-1}$ kg$^{-0.68}$   & &\cite{Pawar2012}	\\
							&						& $10^{-1.77}$ in 3D& m$^2$ s$^{-1}$ kg$^{-1.05} $\\
				$\beta$		& Handling time scaling exponent& 0.75 & - & - & \cite{Pawar2012}\\
				$t_{h, 0}$	& Handling time scaling constant& $10^{3.95}$ in 2D &kg$^{1-\beta}$ s& -& \cite{Pawar2012}	\\
							&						& $10^{3.04}$ in 3D			&kg$^{1-\beta}$ s\\
				\hline
			\end{tabular}
		\end{table}
	\end{centering}

	\nolinenumbers
	
	%%%%%%%%%% Discussion %%%%%%%%%%
\section{Discussion}
	\linenumbers
	
	\nolinenumbers
	
	%%%%%%%%%% Conclusion %%%%%%%%%%
\section{Conclusion}
	\linenumbers
	
	\nolinenumbers
	
	%%%%%%%%%% Bibliography %%%%%%%%%%
	\newpage
	
	\addcontentsline{toc}{section}{\protect\numberline{}References}
	
	\linenumbers
	\printbibliography
	\nolinenumbers
	
	%%%%%%%%%% SI %%%%%%%%%%
	\newpage
	
%to have table and figures numbered from 1 with prefix
\newcommand{\beginsupplement}{%
	\addcontentsline{toc}{section}{\protect\numberline{}Supplementary Information}
	\setcounter{table}{0}
	\renewcommand{\thetable}{S\arabic{table}}%
	\setcounter{figure}{0}
	\renewcommand{\thefigure}{S\arabic{figure}}%
	\setcounter{equation}{0}
	\renewcommand{\theequation}{S\arabic{equation}}
}

\begin{refsection} % for seperate bibliography to main text
\section*{Supplementary Information}
\beginsupplement
%\subsection*{notes}
%need section on value conversions and derivations
%
%move any unreferenced sensitivity analyses here.

%\subsection{Figures}
\subsection{Unit Conversions}
\subsubsection{Functional Response ($ f(\cdot) $)}
	\begin{align}
		\label{Fr_Conversion}
		kg \cdot s^{-1} \cdot 24 \cdot 60 \cdot 60 &= kg \cdot d{-1}
	\end{align}
	
\subsubsection{Metabolic Cost ($ B_m $)}
	Conversion factor for joules to kg wet mass from \citeauthor{weathers2012fundamentals} (\citeyear{weathers2012fundamentals}).
	\begin{align}
		\begin{split}
			\label{Bm_Conversion}
			J \cdot s^{-1} \cdot 24 \cdot 60 \cdot 60 &= J \cdot d{-1}\\
			J \cdot d{-1} \cdot 2.5 \times 10^{-4} &= kg \cdot d{-1}
		\end{split}
	\end{align}
	
\subsection{Growth Curves}
%TODO ensure these are mentioned in the text
	\begin{figure}[H]
	\centering 
	\includegraphics[width=0.7\textwidth]{../../results/pretty_curve}
	\caption{Example of the growth curve and cumulative reproduction expected from a traditional OGM model. Maturation occurs at 1000 days, after which growth is less steep until reaching asymptotic mass.  }
	\label{OGM_Curve}
\end{figure}

\begin{figure}[h]
	\centering
	\includegraphics[width=0.7\textwidth]{../../results/report_growth_curve.pdf}
	\caption{The growth over a fish which consumes in 2D.  Maturation occurs at 5 years (1825 days).  The fish was allowed to shrink by 5\% at the onset of reproduction.}
	\label{growth_curve}
\end{figure}

\clearpage
\subsection{Sensitivity Analysis}
\subsubsection{Maturation Time}
\begin{figure}[H]
	\centering
	\includegraphics[width=\linewidth]{../../results/Sens_High_Resources_Maturation_Time_short_exp1}
	\caption{Effect of maturation time on $c$ and $\rho$ where $\mu = 1$ and resource density is high (100 kg/m$^D$, where $D$ is the dimension).}
	\label{fig:senshighresourcesmaturationtimeshortexp1}
\end{figure}
\begin{figure}[h]
	\centering
	\includegraphics[width=\linewidth]{../../results/Sens_High_Resources_Maturation_Time_short_exp075}
	\caption{Effect of maturation time on $c$ and $\rho$ where $\mu = 0.75$ and resource density is high (100 kg/m$^D$, where $D$ is the dimension).}
	\label{fig:sensmaturationtimeshortexp075}
\end{figure}
\begin{figure}[h]
	\centering
	\includegraphics[width=\linewidth]{../../results/Sens_Low_Resources_Maturation_Time_short_exp1}
	\caption{Effect of maturation time on $c$ and $\rho$ where $\mu = 1$ and resource density is low (0.11 kg/m$^D$, where $D$ is the dimension).}
	\label{fig:senslowresourcesmaturationtimeshortexp1}
\end{figure}
\begin{figure}[h]
	\centering
	\includegraphics[width=\linewidth]{../../results/Sens_Low_Resources_Maturation_Time_short_exp075}
	\caption{Effect of maturation time on $c$ and $\rho$ where $\mu = 0.75$ and resource density is low(0.11 kg/m$^D$, where $D$ is the dimension).}
	\label{fig:senslowresourcesmaturationtimeshortexp075}
\end{figure}
\begin{figure}[h]
	\centering
	\includegraphics[width=\linewidth]{../../results/Sens_Very_Low_Resources_Maturation_Time_short_exp1}
	\caption{Effect of maturation time on $c$ and $\rho$ where $\mu = 1$ and resource density is very low (0.01 kg/m$^D$, where $D$ is the dimension).  At this resource density reproduction can only occur in 3D.}
	\label{fig:sensverylowresourcesmaturationtimeshortexp1}
\end{figure}
\begin{figure}[h]
	\centering
	\includegraphics[width=\linewidth]{../../results/Sens_Very_Low_Resources_Maturation_Time_short_exp075}
	\caption{Effect of maturation time on $c$ and $\rho$ where $\mu = 0.75$ and resource density is very low (0.01 kg/m$^D$, where $D$ is the dimension).  At this resource density reproduction can only occur in 3D.}
	\label{fig:sensverylowresourcesmaturationtimeshortexp075}
\end{figure}




\clearpage
\subsubsection{Metabolic Exponent ($\mu$)}
\begin{figure}[H]
	\centering
	\includegraphics[width=\linewidth]{../../results/Sens_High_Resources_Metabolic_Exponent}
	\caption{Effect of metabolic on $c$ and $\rho$ where resource density is high (100 kg/m$^D$, where $D$ is the dimension)}
	\label{fig:senshighresourcesmetabolicexponent}
\end{figure}
\begin{figure}[h]
	\centering
	\includegraphics[width=\linewidth]{../../results/Sens_Low_Resources_Metabolic_Exponent}
	\caption{Effect of metabolic on $c$ and $\rho$ where resource density is low (0.11 kg/m$^D$, where $D$ is the dimension)}
	\label{fig:senslowresourcesmetabolicexponent}
\end{figure}
\begin{figure}[h]
	\centering
	\includegraphics[width=\linewidth]{../../results/Sens_Very_Low_Resources_Metabolic_Exponent}
	\caption{Effect of metabolic on $c$ and $\rho$ where resource density is very low (0.01 kg/m$^D$, where $D$ is the dimension).  At this resource density reproduction can only occur in 3D.}
	\label{fig:sensverylowresourcesmetabolicexponent}
\end{figure}



\subsubsection{Resource Density}

\begin{figure}[H]
	\centering
	\includegraphics[width=\linewidth]{../../results/Sens_Resource_Density_exp1_broad}
	\caption{Effect of resource density on $c$ and $\rho$ where $\mu = 1$.  Over larger values for resource density.  3D quickly saturates at this density, thus is a nearly straight horizontal line.  See Fig. \ref{fig:sensresourcedensityexp075fine} for detail at lower resource density.  Units are $kg/m^D$, where $D$ is the dimension.}
	\label{fig:sensresourcedensityexp1broad}
\end{figure}
\begin{figure}[h]
	\centering
	\includegraphics[width=\linewidth]{../../results/Sens_Resource_Density_exp075_broad}
	\caption{Effect of resource density on $c$ and $\rho$ where $\mu = 0.75$.  Over larger values for resource density.  There is a lot of numeric instability across resource densities, but the trend appears to be somewhat stable around $\sim$0.8 in 2D and $\sim$0.53 in 3D See Fig. \ref{fig:sensresourcedensityexp075fine} for detail at lower resource density.  Units are $kg/m^D$, where $D$ is the dimension.}
	\label{fig:sensresourcedensityexp075broad}
\end{figure}
	\begin{figure}[h!]
	\centering
	\includegraphics[width=\linewidth]{../../results/Sens_Resource_Density_exp1_fine}
	\caption{Effect of resource density on $c$ and $\rho$ where $\mu = 1$.  Demonstrates the expected trend that under limiting resources the higher scaling of 3D search rate  allows for steeper reproductive scaling (Table \ref{parameters}).  As resources increase and supply shifts more towards being defined by the inverse of handling time, steeper scaling in 2D allows for higher $\rho$ values.  Units are kg/m$^D$, where $D$ is the dimension.}
	\label{fig:sensresourcedensityexp1fine}
\end{figure}





\subsubsection{Proportion of Shrinking Allowed}
\begin{figure}[H]
	\centering
	\includegraphics[width=\linewidth]{../../results/Sens_HighResources_Shrink_exp1}
	\caption{Effect of proportion of shrinking allowed on $c$ and $\rho$ where $\mu = 1$ and resource density is high (100 kg/m$^D$, where $D$ is the dimension).}
	\label{fig:senshighresourcesshrinkexp1}
\end{figure}
\begin{figure}[h]
	\centering
	\includegraphics[width=\linewidth]{../../results/Sens_HighResources_Shrink_exp075}
	\caption{Effect of proportion of shrinking allowed on $c$ and $\rho$ where $\mu = 0.75$ and resource density is high (100 kg/m$^D$, where $D$ is the dimension).}
	\label{fig:senshighresourcesshrinkexp075}
\end{figure}

\begin{figure}[h]
	\centering
	\includegraphics[width=\linewidth]{../../results/Sens_LowResources_Shrink_exp1}
	\caption{Effect of proportion of shrinking allowed on $c$ and $\rho$ where $\mu = 1$ and resource density is low (0.11 kg/m$^D$, where $D$ is the dimension)}
	\label{fig:senslowresourcesshrinkexp1}
\end{figure}
\begin{figure}[h]
	\centering
	\includegraphics[width=\linewidth]{../../results/Sens_LowResources_Shrink_exp075}
	\caption{Effect of proportion of shrinking allowed on $c$ and $\rho$ where $\mu = 0.75$ and resource density is low (0.11 kg/m$^D$, where $D$ is the dimension)}
	\label{fig:senslowresourcesshrinkexp075}
\end{figure}
\begin{figure}[h]
	\centering
	\includegraphics[width=\linewidth]{../../results/Sens_VeryLowResources_Shrink_exp1}
	\caption{Effect of proportion of shrinking allowed on $c$ and $\rho$ where $\mu = 1$ and resource density is very low (0.01 kg/m$^D$, where $D$ is the dimension).  The resource density only allows for reproduction to occur on 3D.}
	\label{fig:sensverylowresourcesshrinkexp1}
\end{figure}
\begin{figure}[h]
	\centering
	\includegraphics[width=\linewidth]{../../results/Sens_VeryLowResources_Shrink_exp075}
	\caption{Effect of proportion of shrinking allowed on $c$ and $\rho$ where $\mu = 0.75$ and resource density is very low (0.01 kg/m$^D$, where $D$ is the dimension).  The resource density only allows for reproduction to occur on 3D.}
	\label{fig:sensverylowresourcesshrinkexp075}
\end{figure}

\clearpage
	\printbibliography
\end{refsection}

\end{document}} % to include word count 

%%%%%%%%%% Formatting %%%%%%%%%%
\usepackage[margin=2cm]{geometry} % margins of 2cm
\linespread{1.5} %1.5 spacing
%\renewcommand{\familydefault}{\sfdefault} % set font to arial clone (helvet)

\usepackage[compact]{titlesec} % reduce spacing bewteen section titles

%%%%%%%%%% Bibliography %%%%%%%%%%
\addbibresource{../Masters_Thesis.bib}


%%%%%%%%%% Document %%%%%%%%%%
\begin{document}
	%%%%%%%%%% Title Page %%%%%%%%%%
	%\newcommand{\crest}{\includegraphics[width = 4cm, keepaspectratio]{../images/IC_Crest.eps}} % Imperial crest
%%formating

\begin{titlepage} % Suppresses headers and footers on the title page
	\includegraphics[width = 7cm, keepaspectratio, left]{../images/imperial_logo}
	\centering % Centre everything on the title page
	
	\scshape % Use small caps for all text on the title page
	
%	\vspace*{\baselineskip} % White space at the top of the page
	
	%------------------------------------------------
	%	Title
	%------------------------------------------------
	
	\rule{\textwidth}{1.6pt}\vspace*{-\baselineskip}\vspace*{2pt} % Thick horizontal rule
	\rule{\textwidth}{0.4pt} % Thin horizontal rule
	
	\vspace{0.75\baselineskip} % Whitespace above the title
	
	{\LARGE The role of resource supply in shaping ontogenetic growth and allocation  in fish\\} % Title
	
	\vspace{0.75\baselineskip} % Whitespace below the title
	
	\rule{\textwidth}{0.4pt}\vspace*{-\baselineskip}\vspace{3.2pt} % Thin horizontal rule
	\rule{\textwidth}{1.6pt} % Thick horizontal rule
	
	\vspace{1\baselineskip} % Whitespace after the title block
	
	%------------------------------------------------
	%	Subtitle
	%------------------------------------------------
	
	%SUBTITLE? % Subtitle or further description
%	Student:
	
	
	\vspace{0.5\baselineskip} % Whitespace before 
	
	{\scshape\Large D\'onal Burns  \\} % my name
	
	\vspace{0.5\baselineskip} % Whitespace below 
	
	\textit{CID: 01749638 \\ Imperial College London \\ Email: donal.burns@imperial.ac.uk} % affiliation and email
	
	\vspace*{2\baselineskip} % Whitespace under the subtitle
	
	
	
%	Supervisor:
%	
%	
%	\vspace{0.5\baselineskip} % Whitespace before 
%	
%	{\scshape\Large Samraat Pawar \\} % supervisor name
%	
%	\vspace{0.5\baselineskip} % Whitespace below 
%	
%	\textit{Imperial College London \\ Email: s.pawar@imperial.ac.uk} % affiliation and email
%	
%	\vspace{3cm} % Whitespace between 
	

	
	%% crest
	
%	\includegraphics[width = 4cm, keepaspectratio]{../images/IC_crest.pdf}
	
	\vspace{0.3\baselineskip} % Whitespace under the Uni logo
	
	Submitted: August 27$^{th}$ 2020 % Publication Date
	
	

		%%Submission clause
	\vspace{3cm}
	
	A thesis submitted in partial fulfilment of the requirements for the degree of
	%Computational Methods in Ecology and Evolution 
	Master of Science at Imperial College London
	\vspace{0.5\baselineskip}
	
	Formatted in the journal style of Functional Ecology	
	\vspace{0.5\baselineskip}
	
	Submitted for the MSc in Computational Methods in Ecology and Evolution
	
\end{titlepage}


	%%%%%%%%%% Abstract %%%%%%%%%%
	\section*{Abstract}
	\linenumbers

	
	
	\nolinenumbers
	%%%%%%%%%% Acknowledgements %%%%%%%%%%
	%\thispagestyle{empty}

\mbox{}\newline\vspace{10mm} \mbox{}\LARGE
%
{\bf Acknowledgements} \normalsize \vspace{5mm}\\
I would like to thank my supervisor Dr. Samraat Pawar as well as fellow lab members Tom Clegg and Olivia Moris for giving me so much of their time on weekly, and on occasion more than weekly, basis.  I would also like to thank Dr. Diego Barneche for his invaluable feedback and Dr. Van Savage for his assistance with some of the initial model development.






	
	%%%%%%%%%% Table of Contents %%%%%%%%%%
	\tableofcontents
	\newpage
	%% start line numbering


	%%%%%%%%%% Introduction %%%%%%%%%%
\section{Introduction}
	\linenumbers
	% Why is growth important
	Understanding how organisms grow and what factors play a role in determining growth
	Larger fish produce more offspring % this is known in general isometrically
	and may even lead to more offspring than if the same mass were spread over two fish \parencite{Barneche2018}.
	Larger fish also use energy more efficiently than multiple smaller ones per unit mass (because they have a lower mass specific metabolic rate \parencite{Peters1983}) so it may actually reduce stress upon the ecosystem from an energetic perspective which could be useful when trying to manage fish stocks %need citations for this
	It is already known that metabolic rates and the size of fish is dependant on temperature and with global warming understanding in greater detail how increased metabolic rates may affect growth is useful.
	
	%
	
	Growth is in essence a balancing act between how much energy an organism can acquire and the amount of energy required for maintenance, that is movement, digestion et cetera.  
	
	%power laws -- needs citation
	Key to understanding rates and their relationship with mass is the concept of power laws. Many biological traits can be described as scaling to the power of some other biological trait.  That is some rate, $Y$, can be expressed for any mass by $Y = Y_0 m^\beta$.  Power laws can be broadly categorised based on the value of their exponent, $\beta$.  Where the exponent does not equal one, the relationship is said to be allometric.  That is the trait does not increase at the same rate as the trait being compared against.  Where the exponent equals one the relationship is described as isometric, that is the two traits increase at the same rate. For the purposes of describing growth these relationships are indispensable.   
	
	Traditionally ontogenetic growth models have relied on knowing how large an organism is expected to grow.
	%	von Bertalanffy / puetter model
	The von Bertalanffy growth equation relies on knowing the longest a fish can be and the length of the fish at the beginning of measurements \parencite{vonBertalanffy1938, Putter1918}.  From here for a known growth rate, the length of the fish after a set amount of time has passed can be predicted.
	%	introduce west 2001 \cite{West2001}
		% then take it to energetics
	One of the best known examples of an OGM is the model developed by \cite{West2001}.  This model is parametrised around the average energy content of tissue asymptotic mass.  Asymptotic mass being the mass at which growth has essentially stopped due to metabolic cost and energy intake equalling each other.  
		%	improved by hou et al \cite{Hou2008} talk about including SDA
		%	
		%	move to including reproduction with \cite{Charnov2001}
	
		% bring up the caveats/assumptions of the model i.e. isometric metabolism and repro, 0.75 scaling of intake. optimal intake at all times etc. test
	% Discuss allometry and isometry here to highlight what scaling super or sub linearly means
	
	In classic OGMs this is most notable in determining how much energy an organism acquires at a given mass where gain = constant $\times$ mass$^{0.75}$.  This scaling is based on the work of \cite{West1997}
	
	In the face two of the key assumptions of \cite{Charnov2001, West2001} OGM not being true, that reproduction and metabolism scale isometrically, there is a need to take a novel approach to modelling fish growth, in particular choosing to focus on developing how intake is described to better reflect the real world.  To do this an obvious starting point is to model intake as a functional response \parencite{Holling1959} so as to reflect real world intake rates.
	% describe a functional response and the work behind it on a general level without equations.
	
	% samraats works feels like it belongs in the methods section more than intro since i have to delve into responses for it to make sense imo
	%	In order to parametrise the functional response for realistic  \cite{Pawar2012} proved invaluable in describing the 
%\subsection{section layout (old)}
%	What is an OGM and ontogenic growth?
%	
%	An ontogenetic model or OGM is a model which describes the development of an organism.  Within a metabolic framework this describes how an organism's assess to and and alocation of energy changes throughout its lifetime.
%	
%	explain west's model
%	
%	explain terms 
%	
%	why is `a` resting metabolic rate?
%	
%	explain charnov's contribution
%	
%	- graph with generic example of the plot for illustration purposes
%	
%	explain what I am adding to `cm` and why --> maybe leave for the methods?
%	
%	talk about barneche 2018
%	
%	raises question of scaling elsewhere 
%	
%	talk about samraats work and how I plan to use it
%	
%	the goal of my paper
	
	\nolinenumbers
	%%%%%%%%%% Methods %%%%%%%%%%
\section{Methods}
	\linenumbers
%	\subsection{Gain}
%	
%	\subsection{Loss}
%	\subsubsection{Maintenance Cost}
%	\subsubsection{Reproductive Cost}
%	\subsection{Reproductive Output}
%	
%	\subsection{notes}
%	c bounded between 0 and 1 since it is basically GSI
	\nolinenumbers
	%%%%%%%%%% Results %%%%%%%%%%
\section{Results}
	\linenumbers
	\subsection{tables and figures}
	
	\begin{centering}

		{\rowcolors{1}{green!80!yellow!50}{red!220!green!220!yellow!220)}}
			
		\begin{table}[h!]
			\label{parameters}
			\caption{Table describing parameters used in the model, along with values units and sources where applicable.}
			\begin{tabular}{c l l l l l}
				\hline
				Parameter 	& Description 			& Value 	& Units 	& Range 		& Source \\
				\hline
				$m$			& Mass					& ?			& kg day$^{-1}$& -			&		\\
				
				$B_m$		& Metabolic Cost		& -			& kg day$^{-1}$& - 			& \cite{Peters1983}\\
				
				$\alpha$	& Age of maturity		& -     	& day		& -				& -\\
				$c$			& Reproduction scaling constant & - & kg day$^{-1}$& 0-1 			& -\\
				$\rho$		& Reproduction scaling exponent	& -	&	-		& 0-1.5			& -\\
				$Z$			& Rate of instantaneous mortality& $2/\alpha$	& & & \cite{Charnov2001}\\
				$k$			& Reproductive senescence & 0.01\\
				
				$\epsilon$	& Resource Conversion Efficiency & 0.70 & - & - 		& \cite{Peters1983} \\
				$X_r$ 		& Resource Density		& -		& kg		& ?				& -\\
				$\gamma$	& Search rate scaling exponent & 0.68 in 2D	& - & - & \cite{Pawar2012} \\
							&						& 1.05 in 3D\\
				$a_0$		& Search rate scaling constant & $10^{-3.08}$ in 2D & m$^2$ s$^{-1}$ kg$^{-0.68}$   & &\cite{Pawar2012}	\\
							&						& $10^{-1.77}$ in 3D& m$^2$ s$^{-1}$ kg$^{-1.05} $\\
				$\beta$		& Handling time scaling exponent& 0.75 & - & - & \cite{Pawar2012}\\
				$t_{h, 0}$	& Handling time scaling constant& $10^{3.95}$ in 2D &kg$^{1-\beta}$ s& -& \cite{Pawar2012}	\\
							&						& $10^{3.04}$ in 3D			&kg$^{1-\beta}$ s\\
				\hline
			\end{tabular}
		\end{table}
	\end{centering}

	\nolinenumbers
	
	%%%%%%%%%% Discussion %%%%%%%%%%
\section{Discussion}
	\linenumbers
	
	\nolinenumbers
	
	%%%%%%%%%% Conclusion %%%%%%%%%%
\section{Conclusion}
	\linenumbers
	
	\nolinenumbers
	
	%%%%%%%%%% Bibliography %%%%%%%%%%
	\newpage
	
	\addcontentsline{toc}{section}{\protect\numberline{}References}
	
	\linenumbers
	\printbibliography
	\nolinenumbers
	
	%%%%%%%%%% SI %%%%%%%%%%
	\newpage
	
%to have table and figures numbered from 1 with prefix
\newcommand{\beginsupplement}{%
	\addcontentsline{toc}{section}{\protect\numberline{}Supplementary Information}
	\setcounter{table}{0}
	\renewcommand{\thetable}{S\arabic{table}}%
	\setcounter{figure}{0}
	\renewcommand{\thefigure}{S\arabic{figure}}%
	\setcounter{equation}{0}
	\renewcommand{\theequation}{S\arabic{equation}}
}

\begin{refsection} % for seperate bibliography to main text
\section*{Supplementary Information}
\beginsupplement
%\subsection*{notes}
%need section on value conversions and derivations
%
%move any unreferenced sensitivity analyses here.

%\subsection{Figures}
\subsection{Unit Conversions}
\subsubsection{Functional Response ($ f(\cdot) $)}
	\begin{align}
		\label{Fr_Conversion}
		kg \cdot s^{-1} \cdot 24 \cdot 60 \cdot 60 &= kg \cdot d{-1}
	\end{align}
	
\subsubsection{Metabolic Cost ($ B_m $)}
	Conversion factor for joules to kg wet mass from \citeauthor{weathers2012fundamentals} (\citeyear{weathers2012fundamentals}).
	\begin{align}
		\begin{split}
			\label{Bm_Conversion}
			J \cdot s^{-1} \cdot 24 \cdot 60 \cdot 60 &= J \cdot d{-1}\\
			J \cdot d{-1} \cdot 2.5 \times 10^{-4} &= kg \cdot d{-1}
		\end{split}
	\end{align}
	
\subsection{Growth Curves}
%TODO ensure these are mentioned in the text
	\begin{figure}[H]
	\centering 
	\includegraphics[width=0.7\textwidth]{../../results/pretty_curve}
	\caption{Example of the growth curve and cumulative reproduction expected from a traditional OGM model. Maturation occurs at 1000 days, after which growth is less steep until reaching asymptotic mass.  }
	\label{OGM_Curve}
\end{figure}

\begin{figure}[h]
	\centering
	\includegraphics[width=0.7\textwidth]{../../results/report_growth_curve.pdf}
	\caption{The growth over a fish which consumes in 2D.  Maturation occurs at 5 years (1825 days).  The fish was allowed to shrink by 5\% at the onset of reproduction.}
	\label{growth_curve}
\end{figure}

\clearpage
\subsection{Sensitivity Analysis}
\subsubsection{Maturation Time}
\begin{figure}[H]
	\centering
	\includegraphics[width=\linewidth]{../../results/Sens_High_Resources_Maturation_Time_short_exp1}
	\caption{Effect of maturation time on $c$ and $\rho$ where $\mu = 1$ and resource density is high (100 kg/m$^D$, where $D$ is the dimension).}
	\label{fig:senshighresourcesmaturationtimeshortexp1}
\end{figure}
\begin{figure}[h]
	\centering
	\includegraphics[width=\linewidth]{../../results/Sens_High_Resources_Maturation_Time_short_exp075}
	\caption{Effect of maturation time on $c$ and $\rho$ where $\mu = 0.75$ and resource density is high (100 kg/m$^D$, where $D$ is the dimension).}
	\label{fig:sensmaturationtimeshortexp075}
\end{figure}
\begin{figure}[h]
	\centering
	\includegraphics[width=\linewidth]{../../results/Sens_Low_Resources_Maturation_Time_short_exp1}
	\caption{Effect of maturation time on $c$ and $\rho$ where $\mu = 1$ and resource density is low (0.11 kg/m$^D$, where $D$ is the dimension).}
	\label{fig:senslowresourcesmaturationtimeshortexp1}
\end{figure}
\begin{figure}[h]
	\centering
	\includegraphics[width=\linewidth]{../../results/Sens_Low_Resources_Maturation_Time_short_exp075}
	\caption{Effect of maturation time on $c$ and $\rho$ where $\mu = 0.75$ and resource density is low(0.11 kg/m$^D$, where $D$ is the dimension).}
	\label{fig:senslowresourcesmaturationtimeshortexp075}
\end{figure}
\begin{figure}[h]
	\centering
	\includegraphics[width=\linewidth]{../../results/Sens_Very_Low_Resources_Maturation_Time_short_exp1}
	\caption{Effect of maturation time on $c$ and $\rho$ where $\mu = 1$ and resource density is very low (0.01 kg/m$^D$, where $D$ is the dimension).  At this resource density reproduction can only occur in 3D.}
	\label{fig:sensverylowresourcesmaturationtimeshortexp1}
\end{figure}
\begin{figure}[h]
	\centering
	\includegraphics[width=\linewidth]{../../results/Sens_Very_Low_Resources_Maturation_Time_short_exp075}
	\caption{Effect of maturation time on $c$ and $\rho$ where $\mu = 0.75$ and resource density is very low (0.01 kg/m$^D$, where $D$ is the dimension).  At this resource density reproduction can only occur in 3D.}
	\label{fig:sensverylowresourcesmaturationtimeshortexp075}
\end{figure}




\clearpage
\subsubsection{Metabolic Exponent ($\mu$)}
\begin{figure}[H]
	\centering
	\includegraphics[width=\linewidth]{../../results/Sens_High_Resources_Metabolic_Exponent}
	\caption{Effect of metabolic on $c$ and $\rho$ where resource density is high (100 kg/m$^D$, where $D$ is the dimension)}
	\label{fig:senshighresourcesmetabolicexponent}
\end{figure}
\begin{figure}[h]
	\centering
	\includegraphics[width=\linewidth]{../../results/Sens_Low_Resources_Metabolic_Exponent}
	\caption{Effect of metabolic on $c$ and $\rho$ where resource density is low (0.11 kg/m$^D$, where $D$ is the dimension)}
	\label{fig:senslowresourcesmetabolicexponent}
\end{figure}
\begin{figure}[h]
	\centering
	\includegraphics[width=\linewidth]{../../results/Sens_Very_Low_Resources_Metabolic_Exponent}
	\caption{Effect of metabolic on $c$ and $\rho$ where resource density is very low (0.01 kg/m$^D$, where $D$ is the dimension).  At this resource density reproduction can only occur in 3D.}
	\label{fig:sensverylowresourcesmetabolicexponent}
\end{figure}



\subsubsection{Resource Density}

\begin{figure}[H]
	\centering
	\includegraphics[width=\linewidth]{../../results/Sens_Resource_Density_exp1_broad}
	\caption{Effect of resource density on $c$ and $\rho$ where $\mu = 1$.  Over larger values for resource density.  3D quickly saturates at this density, thus is a nearly straight horizontal line.  See Fig. \ref{fig:sensresourcedensityexp075fine} for detail at lower resource density.  Units are $kg/m^D$, where $D$ is the dimension.}
	\label{fig:sensresourcedensityexp1broad}
\end{figure}
\begin{figure}[h]
	\centering
	\includegraphics[width=\linewidth]{../../results/Sens_Resource_Density_exp075_broad}
	\caption{Effect of resource density on $c$ and $\rho$ where $\mu = 0.75$.  Over larger values for resource density.  There is a lot of numeric instability across resource densities, but the trend appears to be somewhat stable around $\sim$0.8 in 2D and $\sim$0.53 in 3D See Fig. \ref{fig:sensresourcedensityexp075fine} for detail at lower resource density.  Units are $kg/m^D$, where $D$ is the dimension.}
	\label{fig:sensresourcedensityexp075broad}
\end{figure}
	\begin{figure}[h!]
	\centering
	\includegraphics[width=\linewidth]{../../results/Sens_Resource_Density_exp1_fine}
	\caption{Effect of resource density on $c$ and $\rho$ where $\mu = 1$.  Demonstrates the expected trend that under limiting resources the higher scaling of 3D search rate  allows for steeper reproductive scaling (Table \ref{parameters}).  As resources increase and supply shifts more towards being defined by the inverse of handling time, steeper scaling in 2D allows for higher $\rho$ values.  Units are kg/m$^D$, where $D$ is the dimension.}
	\label{fig:sensresourcedensityexp1fine}
\end{figure}





\subsubsection{Proportion of Shrinking Allowed}
\begin{figure}[H]
	\centering
	\includegraphics[width=\linewidth]{../../results/Sens_HighResources_Shrink_exp1}
	\caption{Effect of proportion of shrinking allowed on $c$ and $\rho$ where $\mu = 1$ and resource density is high (100 kg/m$^D$, where $D$ is the dimension).}
	\label{fig:senshighresourcesshrinkexp1}
\end{figure}
\begin{figure}[h]
	\centering
	\includegraphics[width=\linewidth]{../../results/Sens_HighResources_Shrink_exp075}
	\caption{Effect of proportion of shrinking allowed on $c$ and $\rho$ where $\mu = 0.75$ and resource density is high (100 kg/m$^D$, where $D$ is the dimension).}
	\label{fig:senshighresourcesshrinkexp075}
\end{figure}

\begin{figure}[h]
	\centering
	\includegraphics[width=\linewidth]{../../results/Sens_LowResources_Shrink_exp1}
	\caption{Effect of proportion of shrinking allowed on $c$ and $\rho$ where $\mu = 1$ and resource density is low (0.11 kg/m$^D$, where $D$ is the dimension)}
	\label{fig:senslowresourcesshrinkexp1}
\end{figure}
\begin{figure}[h]
	\centering
	\includegraphics[width=\linewidth]{../../results/Sens_LowResources_Shrink_exp075}
	\caption{Effect of proportion of shrinking allowed on $c$ and $\rho$ where $\mu = 0.75$ and resource density is low (0.11 kg/m$^D$, where $D$ is the dimension)}
	\label{fig:senslowresourcesshrinkexp075}
\end{figure}
\begin{figure}[h]
	\centering
	\includegraphics[width=\linewidth]{../../results/Sens_VeryLowResources_Shrink_exp1}
	\caption{Effect of proportion of shrinking allowed on $c$ and $\rho$ where $\mu = 1$ and resource density is very low (0.01 kg/m$^D$, where $D$ is the dimension).  The resource density only allows for reproduction to occur on 3D.}
	\label{fig:sensverylowresourcesshrinkexp1}
\end{figure}
\begin{figure}[h]
	\centering
	\includegraphics[width=\linewidth]{../../results/Sens_VeryLowResources_Shrink_exp075}
	\caption{Effect of proportion of shrinking allowed on $c$ and $\rho$ where $\mu = 0.75$ and resource density is very low (0.01 kg/m$^D$, where $D$ is the dimension).  The resource density only allows for reproduction to occur on 3D.}
	\label{fig:sensverylowresourcesshrinkexp075}
\end{figure}

\clearpage
	\printbibliography
\end{refsection}

\end{document}