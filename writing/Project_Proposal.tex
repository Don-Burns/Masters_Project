%%%%%%%%%%%%%%%%%%%%%%%%%%%%%%%%%%%%%%%%%
% Formal Book Title Page
% LaTeX Template
% Version 2.0 (23/7/17)
%
% This template was downloaded from:
% http://www.LaTeXTemplates.com
%
% Original author:
% Peter Wilson (herries.press@earthlink.net) with modifications by:
% Vel (vel@latextemplates.com)
%
% License:
% CC BY-NC-SA 3.0 (http://creativecommons.org/licenses/by-nc-sa/3.0/)
% 
% This template can be used in one of two ways:
%
% 1) Content can be added at the end of this file just before the \end{document}
% to use this title page as the starting point for your document.
%
% 2) Alternatively, if you already have a document which you wish to add this
% title page to, copy everything between the \begin{document} and
% \end{document} and paste it where you would like the title page in your
% document. You will then need to insert the packages and document 
% configurations into your document carefully making sure you are not loading
% the same package twice and that there are no clashes.
%
%%%%%%%%%%%%%%%%%%%%%%%%%%%%%%%%%%%%%%%%%

%----------------------------------------------------------------------------------------
%	PACKAGES AND OTHER DOCUMENT CONFIGURATIONS
%----------------------------------------------------------------------------------------

\documentclass[a4paper, 11pt, oneside]{article} % A4 paper size, default 11pt font size and oneside for equal margins



\usepackage[utf8]{inputenc} % Required for inputting international characters
\usepackage[T1]{fontenc} % Output font encoding for international characters
\usepackage{fouriernc} % Use the New Century Schoolbook font
%my packages
\usepackage[comma]{natbib}
\usepackage{graphicx, epstopdf}
\usepackage[mathlines]{lineno}
\usepackage{helvet}
\usepackage[margin=2cm]{geometry} % margins of 2cm
\usepackage{pgfgantt} % for gantt chart
\newcommand{\crest}{\includegraphics[width = 4cm, keepaspectratio]{./images/IC_Crest.eps}} % Imperial crest
%%formating
\linespread{1.5} %1.5 spacing
\renewcommand{\familydefault}{\sfdefault} % set font to arial clone (helvet)

\usepackage[compact]{titlesec} % reduce spacing bewteen section titles




%----------------------------------------------------------------------------------------
%	TITLE PAGE
%----------------------------------------------------------------------------------------

\begin{document} 

\begin{titlepage} % Suppresses headers and footers on the title page

	\centering % Centre everything on the title page
	
	\scshape % Use small caps for all text on the title page
	
	\vspace*{\baselineskip} % White space at the top of the page
	
	%------------------------------------------------
	%	Title
	%------------------------------------------------
	
	\rule{\textwidth}{1.6pt}\vspace*{-\baselineskip}\vspace*{2pt} % Thick horizontal rule
	\rule{\textwidth}{0.4pt} % Thin horizontal rule
	
	\vspace{0.75\baselineskip} % Whitespace above the title
	
	{\LARGE Energy Investment in Growth Rate and Reproduction\\} % Title
	
	\vspace{0.75\baselineskip} % Whitespace below the title
	
	\rule{\textwidth}{0.4pt}\vspace*{-\baselineskip}\vspace{3.2pt} % Thin horizontal rule
	\rule{\textwidth}{1.6pt} % Thick horizontal rule
	
	\vspace{2\baselineskip} % Whitespace after the title block
	
	%------------------------------------------------
	%	Subtitle
	%------------------------------------------------
	
	%SUBTITLE? % Subtitle or further description
		Student:
	
	
	\vspace{0.5\baselineskip} % Whitespace before 
	
	{\scshape\Large D\'onal Burns  \\} % supervisor name
	
	\vspace{0.5\baselineskip} % Whitespace below 
	
	\textit{CID: 01749638 \\ Imperial College London \\ Email: donal.burns@imperial.ac.uk} % affiliation and email
	
	\vspace*{2\baselineskip} % Whitespace under the subtitle
	

	
	Supervisor:
	
	
	\vspace{0.5\baselineskip} % Whitespace before 
	
	{\scshape\Large Samraat Pawar \\} % supervisor name
	
	\vspace{0.5\baselineskip} % Whitespace below 
	
	\textit{Imperial College London \\ Email: s.pawar@imperial.ac.uk} % affiliation and email
	
	\vspace{3cm} % Whitespace between 
	
	%------------------------------------------------
	%	Publisher
	%------------------------------------------------
	
	\crest % Uni logo
	
	\vspace{0.3\baselineskip} % Whitespace under the Uni logo
	
	Submitted 03/04/2020 % Publication Date
	

\end{titlepage}

%---------------------------------------------------------------------------------------
%%more formatting after preamble

\linenumbers

%%%%%%%%%%%%%%%%%%%%%%%%%%%%%%%%%%%%%%%%%%%%%%%%%%%%%%%%%%%%%%%
\subsubsection*{Keywords}
allometry; life history; metabolism; productivity;  reproduction

\section*{Introduction}
Recent results from \cite{Barneche2018} have shown that larger fish produce disproportionately more offspring than smaller fish, that is to say reproductive output is hyper-allometric.  In other words, a single 2kg fish produces more offspring than two 1kg fish.  Currently, many models make the assumption that reproduction is isometric with mass, for example \cite{Charnov2001} and \cite{West2001}.  Additionally, it has been shown that organism resource interactions can also show an allometric relationship based on the dimensionality of interactions, where 3D interactions, such as those in many fish, also showed hyper-allometric scaling \citep{Pawar2012}. 
%This project aims to use and build upon these reproductive and growth models to understand how, from a metabolic standpoint, this phenomenon occurs.
\newline
This project aims to answer the following questions:
\newline
Can allometry in reproduction be incorporated into existing growth rate models which assume an isometric relationship between reproduction and mass? Also, does incorporating this improve accuracy of the model's predictions?
\newline
Does resource interaction dimensionality indicate allometry in reproductive output?
\newline
Can a difference in energy investment towards reproduction be seen between organims with different reproductive strategies, for example r and k selected species?

\section*{Methods}
The project will use a lifetime reproductive output model to infer how energy allocation to reproduction and growth changes throughout development.  Some models will be implemented with some modification so as to take allometric reproduction into account \citep{Charnov2001, West2001}.  Others which already incorporate allometric reproductive output, such as \cite{Burger2019}, will be compared to these modified models for comparison. First, parameters will be optimised so as to maximise reproductive output, then the model will be fitted to data in order to compare how "real world" growth compares to the purely theoretical case and what inferences can be made based on the results.
\newline
The results will then be used to compare the allometry of reproduction with the dimensionality of resource interaction to examine the possibility of a relationship between the two.  

\section*{Anticipated Outcomes}
\begin{itemize}
	\item Design of a model that can describe the growth of organisms, regardless of whether reproduction is allometric or isometric.
	\item  To quantify the energy allocation of an organism throughout ontogeny, specifically with regard to growth and reproduction.
	\item  Determine whether dimensionality of resource interaction may be an indicator of the hypo-  or hyper-allometry in reproduction.
\end{itemize}


\section*{Timeline}


\begin{center}
	\begin{table}[!h]

		\vspace{2mm}
		\begin{tabular}{ p{3cm}   p{10cm} } 
			
			April 15th & Implement currently existing models \\ 
			May 15th & Finish progressing the model / model ready to apply to data\\
			May 15th & Introduction rough draft\\
			May 22nd & Methods rough draft\\
			June 26th & Finish results analysis\\
			July 10th & Results rough draft\\
			August 14th & Hand in full draft to Supervisor\\
			August 27th & Submit thesis\\
	
		\end{tabular}
	\end{table}	
\end{center}
%%%%%%%%%%%%%%%%%%%%%%%%%%%%%%%%%%%%%%%%%%%%%%%%%%%%%%%%%%%%%%%%%%%%%%
%%%%%%%%%%%%%%%%%%%%%%Gant CHART%%%%%%%%%%%%%%%%%%%%%%%%%%%%%%%%%%%%%%
\definecolor{barblue}{RGB}{153,204,254}
\definecolor{groupblue}{RGB}{51,102,254}
\definecolor{linkred}{RGB}{165,0,33}
\definecolor{mainBar}{RGB}{255, 0, 0}
\definecolor{subBar}{RGB}{0, 0, 255}
\renewcommand\sfdefault{phv}
\renewcommand\mddefault{mc}
\renewcommand\bfdefault{bc}
\setganttlinklabel{s-s}{START-TO-START}
\setganttlinklabel{f-s}{FINISH-TO-START}
\setganttlinklabel{f-f}{FINISH-TO-FINISH}
\sffamily
%% my chart
\begin{ganttchart}[
	canvas/.append style={fill=none, draw=black!5, line width=.75pt},
	hgrid style/.style={draw=black!5, line width=.75pt},
	vgrid={*1{draw=black!5, line width=.75pt}},
	title/.style={draw=none, fill=none},
	title label font=\bfseries\footnotesize,
	title label node/.append style={below=7pt},
	include title in canvas=false,
	bar label font=\mdseries\small\color{black!70},
	bar label node/.append style={left=2cm},
	bar/.append style={draw=none, fill=subBar!63},% to change sub bar colour
	bar incomplete/.append style={fill=barblue},
	bar progress label font=\mdseries\footnotesize\color{black!70},
	group incomplete/.append style={fill=groupblue},
	group/.append style = {draw = none, fill = mainBar}, % to change main bar colour
	group left shift=0,
	group right shift=0,
	group height=.5,
	group peaks tip position=0,
	group label node/.append style={left=.6cm},
	group progress label font=\bfseries\small,
	link/.style={-latex, line width=1.5pt, linkred},
	link label font=\scriptsize\bfseries,
	link label node/.append style={below left=-2pt and 0pt}
	]{1}{22}
	\gantttitle[
	title label node/.append style={below left=7pt and -3pt}
	]{WEEKS:\quad1}{1}
	\gantttitlelist{2,...,22}{1} \\
	\ganttgroup{Implement currently existing models}{1}{3} \\
	\ganttgroup{Progress Model}{1}{6} \\
	\ganttgroup{Results Analysis}{8}{12} \\

	\ganttgroup{Writing}{1}{22}\\
%	\ganttbar{Introduction Draft}{1}{6} \\
%	\ganttbar{Methods Draft}{7}{7} \\
%	\ganttbar{Results Draft}{13}{14} \\
%	%\ganttbar{Write remaining sections}{15}{20} \\
%	\ganttbar{Discussion}{15}{20}\\
%	\ganttbar{Conclusion}{15}{20}\\
%	\ganttbar{Abstract}{15}{20}\\
%	\ganttbar{Submit Thesis}{22}{22} \\	

\end{ganttchart}

%%%%%%%%%%%%%%%%%%%%%%%%%%%%%%%%%%%%%%%%%%%%%%%%%%%%%%%%%%%%%%%%%%%%%%
%%%%%%%%%%%%%%%%%%%%%%%%%%%%%%%%%%%%%%%%%%%%%%%%%%%%%%%%%%%%%%%%%%%%%%
\section*{Budget}
\begin{center}
	\begin{table}[!h]
		
		\vspace{2mm}
		\begin{tabular}{ |p{4cm} |p{4cm} |  p{1.5cm}|  p{6.5cm}| } 
			
			\hline
			Category & Item & Cost & Justification\\ 
			\hline \hline
		
			Data Backup and storage& &  & Backup and storage of project data to ensure no lose of time or progress due to data loss\\
			\hline
			&1TB external Hard drive & £62 &\\
			\hline
			Travel & & & Travel to the UK once travel restrictions are lifted\\
			\hline
			&Flight & £100&\\
			\hline

			

		\end{tabular}
	\end{table}	
\end{center}

%%%%%%%%%%%%%%%%%%%%%%%%%%%%%%%%%%%%%%%%%%%%%%%%%%%%%%%%%%%%%%%%%%%%%%
\newpage	
\renewcommand\bibname{References} % set bib Title
\bibliographystyle{agsm}
\bibliography{./Masters_Thesis.bib}



\end{document}
